
In this study, we provide a comprehensive inventory of snoRNAs in fungi
together with an in-depth analysis of the evolution of snoRNA families and
their target specificities. The investigation of 147 different taxa
provides a detailed history of potential gain, loss, and duplication events
for 68 families of box C/D snoRNAs and 50 families of box H/ACA snoRNAs
involving more than 7,800 individual snoRNA sequences. For 18 snoRNA
families previously unrecognized homology with other families has been
uncovered. These data constitute a substantial extension and refinement of
the accumulated knowledge on \sno{}s. Data and refined models will become
available in the \rfam\ database and collectively form an important step
towards a global understanding of the evolution of the snoRNAome. Since our
approach is based on homology search, it is fundamentally limited by the
seed sequences that have been observed and classified as \sno{}s in at
least one organism. It is very unlikely, therefore, that this study
presents a complete picture despite increasing the number of \sno\ sequences
by more than a factor of four. In addition, for 26 of 39 orphan \sno
s (including sequences with single-sequence target predictions only) a
mapping to experimentally verified targets could be found, or at least a
quite convincing prediction based on the Interaction Conservation Index
(ICI) could be assigned.

The processing of this amount of data is well beyond the realm of manual
curation and has been possible only with the help of \snostrip, a pipeline
specifically developed to investigate the evolution of snoRNA families
across a broad phylogenetic range \cite{Bartschat:2014}.  The in-depth
analysis of potential target interactions adds a new layer of
information. We have demonstrated here that the coevolution of \sno s and
their targets can be traced with high resolution based on the functional
characteristics of the \sno s as determined by \snostrip\ together with a
quantitative assessment of predicted RNA-RNA interactions based on the 
Interaction Conservation Index (ICI) \cite{Kehr:2014}.

Similar to Metazoa, fungal \haca s show a higher loss-ratio compared
to \cd s. This might have both a technical and a biological
explanation that manifests itself on two different levels. Since \haca
s do not share long ASEs but rather short bipartite pseudouridylation
pockets, it becomes considerably harder to detect homologous \sno s
over large evolutionary timescales. This effect may limit the scope of
the homology search procedure. The short interacting regions make
these molecules also more vulnerable to mutations that disrupt the
snoRNA-target interaction. At the same time, the presence of the
second, independent ASE in the other hairpin may be a sufficient cause
to retain mutated genes.

In general, fungal \sno s have well-preserved target interactions, and
most families are found to contain exactly one highly conserved
anti-sense element. The remaining target region is in turn free to
evolve or to adapt to new lineage-specific or even species-specific
targets. Here, we introduced a variation on the ICI measure adapted to
subclades, allowing a much more detailed quantitative assessment of
target turnover. Many of the predictions made here, of course, await
experimental validation, given that experimental evidence for
RNA-target interactions as well as direct measurements of chemical
modifications in the primary target molecules (rRNAs and snRNAs) are
still restricted to a few model organisms.

The computational analysis reported here strongly suggests that snoRNAs not
only address a highly conserved ASE but also frequently have additional,
secondary targets. The possibility that a single \sno\ target site exerts
two distinct guiding functions has been exemplarily reported for budding
yeast \haca s. The budding yeast \sno\ family snR3 (HACA\_3), for example,
is verified to target two modification sites in its second hairpin
\cite{Schattner:2004}.  Both interactions can be observed across Dikarya.
Nevertheless, there is still very little experimental data on the
generality of this effect, and most of the predicted 'double' target sites
will still require experimental verification. Convincing examples of
remarkably conserved multiple interactions are found in \cd\ families snR40
(CD\_43) and snR70 (CD\_61), which exhibit two and five high-scoring
target-interactions at a single ASE, respectively.  These findings suggest
the possibility that \sno s are, at least under certain circumstances, able
to guide different modifications with the same ASE. This might be dependent
on developmental states, or more complex mechanisms involving
conformational changes of the target.

In some cases of \haca s, these additional targets exhibit better ICI
scores than the annotated modification sites. Since the ICI combines
evidence from thermodynamic stability and evolutionary conservation, these
predictions cannot be easily dismissed as false positives. The specialized
ribosome hypothesis proposes distinct ribosomal conformations in different
developmental stages and stress levels that might also entail different
chemical modification patterns of the rRNAs; it is entirely plausible, in
this scenario, that some modifications and, thus, snoRNA interaction sites
have remained undetected \citep{Xue:2012}. The existence of stress-induced
conditional pseudouridylations indeed has been reported for the U2
snRNA of budding yeast \cite{Wu:2011}. The snR81 RNA, which is also
responsible for the guidance of a constitutive U2 pseudouridylation, guides
one of the novel modifications through imperfect and redundant base
pairing. The authors speculate that conditionally induced modifications in
RNA may well be a rather frequent phenomenon.

We also found convincing evidence that some modifications are guided by
two, three, or even more \sno\ families. First, this includes redundant
guides, meaning that two snoRNA families of the same species are
responsible for the same modification. Second, we observed several target
sites that are addressed by different snoRNA families in different
taxonomic groups. A good example of the latter situation is the predicted
pseudouridine at position 5.8S-18. Although there is no direct experimental
evidence that this particular position is modified \emph{in vivo}, the site
is predicted as a target for several distinct snoRNA families by \snoop\
(see Supplementary Section S21).

The fact that specific modification sites are predicted to be guided
by more than just one \sno\ family in the same organism has several
possible reasons. SnoRNA expression was recently reported to be
strongly regulated in development and between tissues or cell lines
\cite{Kapushesky:2012, Jorjani:2016}. It may thus be necessary for the
organism to compensate for snoRNAs that are lowly expressed under
certain circumstances to maintain the functional modification levels
of the target RNA. This may be achieved through paralogous snoRNAs or
redundant target binding capabilities of other \sno\ families.

In summary, we observe that the landscape of snoRNAs keeps constantly
changing in the kingdom of Fungi. We observe both the extinction of
entire \sno\ families and the innovation of new ones. Even the
function of snoRNA families itself changes at these evolutionary
scales, showing loss, gain, and turn-over of guiding functions that
lead to target switches. The number of known snoRNA families in Fungi
is lower than in animals, correlating well with the observation that
animals have more (reported) modification sites in their rRNAs and
snRNAs than ``lower'' Eukaryotes (see \texttt{Modomics} and the
\texttt{RNA Modification Database} \cite{Machnicka:2013,
Cantara:2011}) or even Bacteria (which have target-specific enzymes
for each individual modification instead of the generic enzyme
machinery with snoRNAs as evolutionary flexible ``address labels'').

There are many similarities between the fungal and the metazoan snoRNAome.
A common feature is a detectable burst in the \sno\ diversity at each major
branching point in the taxonomic tree of both kingdoms.  In case of fungal
\cd s, the distribution of orphan, single guided, and double guided \snos\
is quite similar compared to animals, as reported by the human \sno\ atlas
\citep{Jorjani:2016}: over 75\% of the box C/D \sno s are found to be
single guided (over 70\% in Metazoa).  In both human and fungi the
remainder is about equally distributed among double guided and orphan
snoRNAs. The situation is somewhat different for \haca s: in human double
guided \sno s comprise the largest group (47\%), while in Fungi, only 22\%
of the \haca\ families target two distinct pseudouridylation sites with
both hairpins.

